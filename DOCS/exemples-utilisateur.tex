%========================================================================
%
%========================================================================
\section{Des exemples}

   Le répertoire {\tt examples} des sources de {\sc ndes} comprend un
certain nombre d'exemples. Certains de ces exemples servent de
tutoriels et ont donc été décrits précédemment. D'autres sont
présentés ici.

%------------------------------------------------------------------------
%
%------------------------------------------------------------------------
\subsection{Utilisation des sondes}

\subsubsection{Mesurer un débit}

   Le programme d'exemple {\tt debits.c} montre plusieurs méthodes de
calcul d'un débit.

   Considérons le cas simple d'une source, dont nous voulons mesurer
le débit de sortie. Pour cela, nous allons insérer une sonde sur la
taille des paquets transmis grâce à la méthode {\tt
  PDUSource\_setPDUGenerationSizeProbe}.

   Le type de sonde dépendra de la mesure souhaitée.

%.......................................................................
%
%.......................................................................
\paragraph{Débit moyen}

   Supposons que nous voulons simplement connaître le débit moyen sur
toute la transmission, alors une sonde mesurant la moyenne sera parfaitement
suffisante :

%.......................................................................
%
%.......................................................................
\paragraph{Débit ``instantanté''}

