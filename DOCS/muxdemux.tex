%========================================================================
%
%========================================================================
\section{Un multiplexeur/démultiplexeur}
\label{section:muxdemux}

   Un outil de multiplexage/démultiplexage élémentaire est implanté
dans {\tt src/muxdemux.c} définissant une entité de multiplexage et
une entité de démultiplexage. Le but est de fournir un cadre simple
mais propre permettant à l'utilisateur de multiplexer dans une même
séquence d'entités des flux différents et d'\^etre capable de les
différencier à l'arrivée.

   Attention, cet outil ne fournit aucune fonctionnalité de stockage,
si bien que si le premier élément de la séquences séparant le
multiplexeur du démultiplexeur n'est pas assez rapide, alors des {\sc
  pdu} peuvent être perdues.


%------------------------------------------------------------------------
%
%------------------------------------------------------------------------
\subsection{Multiplexeur : déclaration et initialisation}

%------------------------------------------------------------------------
%
%------------------------------------------------------------------------
\subsection{Démultiplexeur : déclaration et initialisation}

%------------------------------------------------------------------------
%
%------------------------------------------------------------------------
\subsection{Utilisation d'un filtre pour éviter le multiplexage}

   Il peut être intéressant de conditionner des traitements sur
l'identifiant de multiplexage sans démultiplexer les flux. Un filtre
peut être simplement construit pour cela. Les filtres sont décrits
dans la section \ref{section:filtres}.

\index{muxDemuxSender\_createFilterFromSAP}
\begin{verbatim}
struct PDUFilter_t * muxDemuxSender_createFilterFromSAP(struct muxDemuxSenderSAP_t * sap);
\end{verbatim}

   Un exemple est disponible dans le fichier de test {\tt tests/rr-mux.c}.
