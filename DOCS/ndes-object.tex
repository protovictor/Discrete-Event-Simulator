%========================================================================
%
%========================================================================
\section{Le type ndesObject}

   Le type {\tt struct ndesObject\_t} est défini dans {\tt ndesObject.h}.

%------------------------------------------------------------------------
%
%------------------------------------------------------------------------
\subsection{Déclaration et initialisation}

   Un objet qui doit aussi être un ndesObject devra utiliser la macro
   \lstinline{declareAsNdesObject}, par exemple de la façon suivante


\index{declareAsNdesObject}
\begin{verbatim}
 struct monType_t {
   declareAsNdesObject; //< C'est un ndesObject 
   ...
}
\end{verbatim}

   Un certain nombre de fonctions sont définies automatiquement grâce
à la macro \lstinline{defineObjectFunctions}. Cette dernière s'utilise
de la façon suivante

\index{defineObjectFunctions}
\begin{verbatim}
/**
 * @brief Définition des fonctions spécifiques liées au ndesObject
 */
defineObjectFunctions(monType);
\end{verbatim}

   Le type doit alors lui-même être déclaré en temps que ndesObject au
travers d'une structure de type \lstinline{struct ndesObjectType\_t}
définissant les principales fonctions de manipulation. La macro
\lstinline{ndesObjectTypeDefaultValues} permet d'utiliser les
fonctions de base, qui sont généralement suffisantes.

   Finalement, la déclaration se fait donc de la façon suivante
   
\index{ndesObjectTypeDefaultValues}
\begin{verbatim}
/**
 * @brief Mes objets sont aussi des ndesObject
 */
struct ndesObjectType_t monTypeType = {
  ndesObjectTypeDefaultValues(monType)
};
\end{verbatim}

   Enfin, lors de l'initialisation d'un objet, il est indispensable
d'initialiser son ``côté objet'' grâce à la fonction
\lstinline!ndesObjectInit! :

\index{ndesObjectInit}
\begin{verbatim}
   struct monType_t * t;
   ...
   ndesObjectInit(t, monType);
   ...
\end{verbatim}

%------------------------------------------------------------------------
%
%------------------------------------------------------------------------
\subsection{Les fonctions de manipulation d'un type d'objets}

%------------------------------------------------------------------------
%
%------------------------------------------------------------------------
\subsection{Les listes d'objets}

   Les objets utilisant  les sont définies dans {\tt ndesObjectFile.h}.

%........................................................................
%
%........................................................................
\subsubsection{Création}

   Une liste est créée par un appel à la fonction suivante

\index{ndesObjectFile\_create}
\begin{verbatim}
struct ndesObjectFile_t * ndesObjectFile_create(struct ndesObjectType_t * t);
\end{verbatim}

   L'unique paramètre est le type des objets décrit par une
\lstinline{struct ndesObjectType\_t} initialiseé comme décrit
précédemment.

%........................................................................
%
%........................................................................
\subsubsection{Parcours d'une file}

   L'utilisation d'une liste nécessite régulièrement d'en parcourir
les éléments, par exemple pour appliquer une fonction à chacun d'entre
eux. La notion d'{\em itérateur} est là pour ça, \ldots

   Un itérateur est un objet de type \lstinline{struct ndesObjectFileIterator\_t}.
 Il est initialisé par la fonction
suivante 

\index{ndesObjectFile\_createIterator}
\begin{verbatim}
struct ndesObjectFileIterator_t * ndesObjectFile_createIterator(struct
ndesObjectFile_t * of);
\end{verbatim}

   Le parcours est alors réalisé par des appels successifs à la
fonction \lstinline{ndesObjectFile\_iteratorGetNext} :

\index{ndesObjectFile\_iteratorGetNext}
\begin{verbatim}
struct ndesObjectFile_t * ndesObjectFile_iteratorGetNext(struct ndesObjectFileIterator_t * ofi);
\end{verbatim}

   Cette fonction renvoie, à chaque appel, un pointeur sur l'élément
suivant de la liste. Lorsque la liste a été intégralement parcourue,
elle renvoie un pointeur {\tt NULL}.

   Finalement, un itérateur doit être détruit par un appel à la
fonction \lstinline{ndesObjectFile\_deleteIterator}

\index{ndesObjectFile\_deleteIterator}
\begin{verbatim}
void ndesObjectFile_deleteIterator(struct ndesObjectFileIterator_t * ofi);
\end{verbatim}
