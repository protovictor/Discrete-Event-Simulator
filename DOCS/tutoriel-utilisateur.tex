%========================================================================
%
%========================================================================
\section{Tutoriel}
\label{section:user-tuto}

%------------------------------------------------------------------------
\subsection{Introduction}

   Félicitations ! Tu t'apprètes à entrer dans le monde fabuleux de
{\sc ndes}, \ldots

   Il s'agit en gros d'une sorte de librairie astucieusement
agrémentée de nombreux bugs dont le but et de t'aider (ou pas) à faire
ton propre simulateur réseau !

   L'implantation d'un simulateur passera donc par l'écriture d'un
programme C utilisant cette librairie.

%------------------------------------------------------------------------
\subsection{Installation}

\begin{verbatim}
 $ git clone https://github.com/Manu-31/ndes.git
 $ cd ndes
 $ make
 $ make install
\end{verbatim}

   Pas de panique, ça n'installe rien pour le moment ! Cela ne fait
que copier une librairie (statique) dans un répertoire dédié dans
l'arborescence.

   On peut également compiler quelques programmes de tests de non
régression :

\begin{verbatim}
 $ make tests-bin
\end{verbatim}
%$
   Et pourquoi pas les lancer :

\begin{verbatim}
 $ make tests
\end{verbatim}
%$

   Évidemment, les messages {\tt OK} ne prouvent rien ! Seuls les
messages d'erreur prouvent qu'il y a un problème, \ldots {} comme s'il
y avait besoin de preuves.

%------------------------------------------------------------------------
\subsection{Ma première simulation : chouette, une file M/M/1 !}

   Le fichier source (et son Makefile, parce qu'on ne se moque pas du
client) se trouve dans le répertoire {\tt example/tutorial-1}.

   Dans {\sc ndes}, le système va être modélisé par une source, suivie
d'une file d'attente, en aval de laquelle se trouve un serveur suivi
par un puits. Je te laisse faire un dessin et je publierai ici le plus
joli !

%........................................................................
\subsubsection{Création du simulateur}

   Avant toute manipulation, on crée le simulateur de la façon suivante

\begin{verbatim}
#include <motsim.h>

...

   /* Creation du simulateur */
   motSim_create();
\end{verbatim}

%........................................................................
\subsubsection{Création du puits}

   Un puits sera un objet qui reçoit sans rechigner des messages et
qui les détruit instantanément. On le crée très simplement de la façon
suivante

\begin{verbatim}
#include <pdu-sink.h>

...

   struct PDUSink_t       * sink; // Déclaration d'un puits

   ...

   /* Crétion du puits */
   sink = PDUSink_create();
\end{verbatim}

%........................................................................
%
%........................................................................
\subsubsection{Création du serveur}

   Dans la mesure où on ne souhaite rien faire de bien intelligent
avec notre serveur, nous allons utiliser un serveur générique, offert
par la maison, qui ne recule devant aucun sacrifice !

   Il s'utilise comme ça 

\begin{verbatim}
#include <srv-gen.h>

...

   struct srvGen_t        * serveur; // Déclaration d'un serveur générique

...

   /* Création du serveur */
   serveur = srvGen_create(sink, (processPDU_t)PDUSink_processPDU);
\end{verbatim}

   La création du serveur est un peu plus compliquée que celle du
puits. La raison est que le serveur doit savoir à qui envoyer les
clients (des {\sc pdu}s pour {\sc ndes}) après les avoir servis. Il
faut donc lui dire quelle fonction leur appliquer (ici {\tt
  PDUSink\_processPDU} un fonction spécifiques aux puits) et à quelle
entité cette fonction s'applique (notre unique puits, ici). Pour
connaître   la fonction, il faut la chercher dans la description de
l'entité cible. Dans notre exemple, la fonction {\tt
  PDUSink\_processPDU} est décrite dans la section dédiée aux puits.

   Plus de précisions sur cette histoire de fonctions dans la section
\ref{section:architecture} de description de l'architecture générale.

   Comme nous voulons une file M/M/1, nous devons dire à notre serveur
que son temps de traitement est exponentiel de paramètre {\tt mu}. Et
zou :

\begin{verbatim}
   float   mu = 10.0; // Paramètre du serveur

   ...

   /* Paramétrage du serveur */
   srvGen_setServiceTime(serveur, serviceTimeExp, mu);
\end{verbatim}

   Les serveurs génériques sont décrits plus précisément dans la
section \ref{section:srv_gen}.

%........................................................................
%
%........................................................................
\subsubsection{Création de la file}

   Une file permet de stocker des {\sc pdu} en transit. On la
construit ainsi

\begin{verbatim}
   struct filePDU_t       * filePDU; // Déclaration de notre file

    ...

   /* Création de la file */
   filePDU = filePDU_create(serveur, (processPDU_t)srvGen_processPDU);
\end{verbatim}

   Sans autre forme de procès, les files ne sont pas bornées. Elles
sont décrites plus précisément dans la section \ref{section:file}. Les
deux paramètres de la fonction de création ont le même rôle que ceux
de la fonction de création du serveur.

%........................................................................
%
%........................................................................
\subsubsection{Création de la source}

   Nous voilà à la source ! Nous allons utiliser un objet de {\sc
ndes} dont le rôle est de produire des {\sc pdu}s. Mais avant cela,
nous devons créer un autre objet qui lui indiquera les dates
auxquelles les produire : il s'agit d'un ``générateur de date'',
original, non ? Les générateurs de dates sont décrits dans la section \ref{section:date_generator}.

   On veut une source poisonnienne, donc un générateur exponentiel :

\begin{verbatim}
#include <date-generator.h>

...

   struct dateGenerator_t * dateGenExp; // Un générateur de dates
   float   lambda = 5.0 ; // Intensité du processus d'arrivée

   ...

   /* Création d'un générateur de date */
   dateGenExp = dateGenerator_createExp(lambda);
\end{verbatim}

   Et maintenant nous pouvons donc créer notre source :

\begin{verbatim}
#include <pdu-source.h>

...

   struct PDUSource_t     * sourcePDU;  // Une source

   ...

   sourcePDU = PDUSource_create(dateGenExp, 
				filePDU,
				(processPDU_t)filePDU_processPDU);
\end{verbatim}

   Le premier paramètre est donc l'objet qui lui permet de déterminer
les dates d'envoi. Les deux suivants sont similaires à ceux passés
lors de la cration de la file et du serveur.
   
%........................................................................
%
%........................................................................
\subsubsection{Mise en place de sondes}

   Oui, je sais, le titre fait un peu peur, mais ça va bien se
passer. Lorsqu'on veut lancer un simulateur, on espère en général
obtenir des résulats. Dans {\sc ndes}, ceux-ci seront collectés par un
outil spécifique, la sonde.

   Les différents types de sonde sont décrits dans la section
\ref{section:sondes}. Nous utiliserons ici uniquement des sondes
exhaustives. Pour chaque objet décrit, la liste des sondes
disponibles est fournie. 

   Nous les déclarons et initialisons comme ça :

\begin{verbatim}
#include <probe.h>

...

   struct probe_t         * sejProbe, * iaProbe, * srvProbe; // Les sondes

...

   /* Une sonde sur les interarrivées */
   iaProbe = probe_createExhaustive();
   dateGenerator_setInterArrivalProbe(dateGenExp, iaProbe);

   /* Une sonde sur les temps de séjour */
   sejProbe = probe_createExhaustive();
   filePDU_addSejournProbe(filePDU, sejProbe);

   /* Une sonde sur les temps de service */
   srvProbe = probe_createExhaustive();
   srvGen_setServiceProbe(serveur, srvProbe);
\end{verbatim}

%........................................................................
%
%........................................................................
\subsubsection{Lancement de la simulation}

   Ça y est, nous y sommes enfin, voici comment démarrer la
simulation. Nous devons activer les entités voulue au moment
voulu. Ici, il n'y a que l'unique source à activer, et nous souhaitons
le faire dès le début de la simulation :

\begin{verbatim}
   /* On active la source */
   PDUSource_start(sourcePDU);
\end{verbatim}

   Nous lançons maintenant la simulation. Nous allons la faire durer
100 secondes :

\begin{verbatim}
   /* C'est parti pour 100 000 millisecondes de temps simulé */
   motSim_runUntil(100000.0);
\end{verbatim}

   Nous pouvons ensuite afficher quelques paramètres internes du
simulateur :

\begin{verbatim}
   motSim_printStatus();
\end{verbatim}

   Et voilà !

%........................................................................
%
%........................................................................
\subsubsection{Affichage des résultats}

   Maintenant que notre simulation est terminée, on a certainement 
envie d'en voir le résultat. On utilisera pour cela des fonctions
founies par les sondes, par exemple :

\begin{verbatim}
   /* Affichage de quelques résultats scalaires */
   printf("%d paquets restant dans  la file\n",
	  filePDU_length(filePDU));
   printf("Temps moyen de sejour dans la file = %f\n",
	  probe_mean(sejProbe));
   printf("Interarive moyenne     = %f (1/lambda = %f)\n",
	  probe_mean(iaProbe), 1.0/lambda);
   printf("Temps de service moyen = %f (1/mu     = %f)\n",
	  probe_mean(srvProbe), 1.0/mu);
\end{verbatim}

   Génial, non ? Non ! Mais la suite est plus rigolote, \ldots

%........................................................................
%
%........................................................................
\subsubsection{Tracé de courbes}

   Pour obtenir des résultats plus riches, nous allons utiliser
(depuis le simulateur) un affichage {\em gnuplot}. Nous avons ici au
moins deux courbes intéressantes à tracer, donc nous allons écrire une
 fonction pour cela :

\begin{verbatim}

/*
 * Affichage (via gnuplot) de la probre pr
 * elle sera affichée comme un graphbar de nbBar barres
 * avec le nom name
 */
void tracer(struct probe_t * pr, char * name, int nbBar)
{
   struct probe_t   * gb;
   struct gnuplot_t * gp;

   /* On crée une sonde de type GraphBar */
   gb = probe_createGraphBar(probe_min(pr), probe_max(pr), nbBar);

   /* On convertit la sonde passée en paramètre en GraphBar */
   probe_exhaustiveToGraphBar(pr, gb);

   /* On la baptise */
   probe_setName(gb, name);

   /* On initialise une section gnuplot */
   gp = gnuplot_create();

   /* On recadre les choses */
   gnuplot_setXRange(gp, probe_min(gb), probe_max(gb)/2.0);

   /* On affiche */
   gnuplot_displayProbe(gp, WITH_BOXES, gb);
}
\end{verbatim}

   L'utilisation de {\em gnuplot} est décrite dans la section
\ref{section:gnuplot}.

   Attention à ne pas oublier de mettre une petite pause à la fin de
notre programme principal, sinon il s'arrète et il tue ses processus
fils, et donc l'affichage {\rm gnuplot} disparaît.

%........................................................................
%
%........................................................................
\subsubsection{Utilisation de notre premier simulateur}

   Il ne nous reste plus qu'à compiler notre programme et à le
lancer. Le répertoire {\tt examples/tutorial-1} contient également un
makefile que je te laisse observer, mais en gros, il faut aller
chercher les includes et la librairie. En utilisant ce {\tt Makefile},
on a donc :

\begin{verbatim}
 $ cd examples/tutorial-1
 $ make
 $ ./mm1
[MOTSI] Date = 99999.846555
[MOTSI] Events : 998243 created (3 m + 998240 r)/998242 freed
[MOTSI] Simulated events : 998243 in, 998242 out, 1 pr.
[MOTSI] PDU : 998242 created (27 m + 998215 r)/998242 freed
[MOTSI] Total malloc'ed memory : 25169976 bytes
[MOTSI] Realtime duration : 1 sec
0 paquets restant dans  la file
Temps moyen de sejour dans la file = 0.061008
Interarive moyenne     = 0.200352 (1/lambda = 0.200000)
Temps de service moyen = 0.100176 (1/mu     = 0.100000)
*** ^C pour finir ;-)
 $
\end{verbatim}

%------------------------------------------------------------------------
%
%------------------------------------------------------------------------
\subsection{Ma deuxiéme simulation : super, encore une file M/M/1 !}

   Le premier exemple est sympa, vas-tu me dire, mais ce serait mieux
si le temps de traitement dépendait de la taille des clients (comme
des paquets dont le temps d'émission dépend de la taille !)

   Et bien soit ! C'est ce que nous allons faire dans ce deuxième
tutoriel. Les fichiers se trouvent dans le répertoire {\tt
example/tutorial-2}.

   Comme je n'ai pas que ça à faire, et toi non plus probablement, je
vais juste commenter les différences par rapport au tutoriel
précédent.

   Une petite modification cosmétique est apportée dans la définition
des paramètres de la simulation, que l'on va présenter avec un
vocabulaire plus réseau et moins file d'attente :

\begin{verbatim}
   float frequencePaquets = 5.0;      // Nombre moyen de pq/s
   float tailleMoyenne    = 1000.0;   // Taille moyenne des pq
   float debit            = 10000.0;  // En bit par seconde
\end{verbatim}

   Ces valeurs sont peu réalistes, mais choisies astucieusement pour
avoir les mêmes résultats que le premier tutoriel.

%........................................................................
%
%........................................................................
\subsubsection{Génération de la taille des paquets}

   Voilà la principale différence avec le premier tutoriel. On va
utiliser un générateur de nombres aléatoires pour avoir des paquets de
taille variable :

\begin{verbatim}
   /* Création d'un générateur de taille (tailles non bornées) */
   sizeGen = randomGenerator_createUInt();
   randomGenerator_setDistributionExp(sizeGen, 1.0/tailleMoyenne);

   /* Affectation à la source */
   PDUSource_setPDUSizeGenerator(sourcePDU, sizeGen);
\end{verbatim}

  Les générateurs de nombres aléatoires sont décrits dans la section
\ref{section:rand-gen}. Profitons en pour placer une sonde sur cette
taille, afin de vérifier qu'elle a bien la bonne tête :

\begin{verbatim}
   /* Une sonde sur les tailles */
   szProbe = probe_createExhaustive();
   randomGenerator_setValueProbe(sizeGen, szProbe);
\end{verbatim}

   Avec ça, on pourra tracer une jolie courbe de plus !

%........................................................................
%
%........................................................................
\subsubsection{Prise en compte par le serveur}

   Ce que l'on vient de faire ne sert à rien si le serveur ne le prend
pas en compte. Il nous faut donc préciser que ce dernier sert chaque
client en un temps dépendant de la taille :

\begin{verbatim}
   /* Paramétrage du serveur */
   srvGen_setServiceTime(serveur, serviceTimeProp, 1.0/debit);
\end{verbatim}

   Cela signifie que le temps de service d'un client est proportionnel
à sa taille avec comme coefficient l'inverse du débit.

