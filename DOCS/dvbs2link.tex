%------------------------------------------------------------------------
%
%------------------------------------------------------------------------
\section{Un lien {\sc dvb-s2}}

   Le module {\tt dvb-s2-ll} permet de simuler une liaison de type
DVB-S2.

\subsection{Création}

Une telle entité sera créée grâce à la fonction suivante

\index{DVBS2ll\_create}
\begin{verbatim}
struct DVBS2ll_t * DVBS2ll_create(void * destination,
				  processPDU_t destProcessPDU,
				  unsigned long  symbolPerSecond,
				  unsigned int FECFrameBitLength);
\end{verbatim}

   Les caractéristiques principales du canal sont passées en
   paramètres.
Pour simplifier le
paramètrage, les macros suivantes sont fournies

\index{FEC\_FRAME\_BITSIZE\_LARGE}
\begin{verbatim}
#define FEC_FRAME_BITSIZE_LARGE 64800
\end{verbatim}

\subsection{Ajout/modification d'un MODCOD}

   À un canal DVB-S2 doit être associé un certain nombre de MODCODs
disponibles. Cela se fait par la fonction suivante

\index{DVBS2ll\_addModcod}
\begin{verbatim}
int DVBS2ll_addModcod(struct DVBS2ll_t * dvbs2ll,
		      unsigned int bbframeBitLength,
		      unsigned int bitsPerSymbol);
\end{verbatim}

   La valeur retourné est l'indice du MODCOD. Pour simplifier le
paramètrage, les macros suivantes sont fournies

\index{C14SIZE}
\index{C13SIZE}
\index{C25SIZE}
\index{C12SIZE}
\index{C35SIZE}
\index{C23SIZE}
\index{C34SIZE}
\index{C45SIZE}
\index{C56SIZE}
\index{C89SIZE}
\index{C910SIZE}
\index{MQPSK}
\index{M8PSK}
\index{M16APSK}
\index{M32APSK}
\begin{verbatim}
#define  C14SIZE  16008
#define  C13SIZE  21408
#define  C25SIZE  25728
#define  C12SIZE  32208
#define  C35SIZE  38688
#define  C23SIZE  43040
#define  C34SIZE  48408
#define  C45SIZE  51648
#define  C56SIZE  53840
#define  C89SIZE  57472
#define  C910SIZE 58192

#define   MQPSK    2
#define   M8PSK    3
#define   M16APSK  4
#define   M32APSK  5
\end{verbatim}

   Si l'on souhaite modifier le MODCOD n, par exemple pour refléter un
chagenement de conditions de réception, on pourra utiliser la fonction

\index{DVBS2ll\_setModcod}
\begin{verbatim}
void DVBS2ll_setModcod(struct DVBS2ll_t * dvbs2ll,
                       int n,
		       unsigned int bbframeBitLength,
		       unsigned int bitsPerSymbol);
\end{verbatim}

\subsection{Exemple}

   À titre d'exemple, voici le code permettant de créer un lien DVB-S2
(menant par exemple vers un puits) de 10Msymboles/sec avec 4 MODCODs :

\begin{verbatim}
struct DVBS2ll_t * dvbS2;

...

dvbS2 = DVBS2ll_create(sink,
                       (processPDU_t)PDUSink_processPDU,
                       FEC_FRAME_BITSIZE_LARGE,
                       10000000);
DVBS2ll_addModcod(dvbS2, C13SIZE, MQPSK);
DVBS2ll_addModcod(dvbS2, C25SIZE, M8PSK);
DVBS2ll_addModcod(dvbS2, C34SIZE, M16APSK);
DVBS2ll_addModcod(dvbS2, C56SIZE, M16APSK);
\end{verbatim}

