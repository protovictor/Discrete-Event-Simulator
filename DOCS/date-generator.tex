%========================================================================
%
%========================================================================
\section{Les générateurs de dates}
\label{section:date-gen}

   Les générateurs de dates servent à générer \ldots{} des dates ! Ils
seront donc utiles en particulier pour les sources de {\sc pdu}
auxquelles ils fourniront les dates de création des fameuses {\sc
  pdu}.

   Un générateur de date est alors essentiellement caractérisé par une
loi d'inter-arrivée. Il s'agit de la loi qui régit le temps séparant
deux dates consécutives.

%------------------------------------------------------------------------
%
%------------------------------------------------------------------------
\subsection{Création}

   Un générateur aléatoire est créé de la façon suivante

\index{dateGenerator\_create}
\begin{verbatim}
struct dateGenerator_t * dateGenerator_create();
\end{verbatim}

   Il est alors inutilisable ! Il doit en effet lui être associée une
loi d'inter-arrivée. Cela dit, il existe des constructeurs (décrits
ci-dessous) permettant de créer un générateur de date directement
initialisé avec un générateur aléatoire.

%------------------------------------------------------------------------
%
%------------------------------------------------------------------------
\subsection{Démarrage}

   Le démarrage d'un générateur de date est la date à partir de
laquelle il est susceptible de produire des dates (je ne parle pas de
dattes hein !).

   Elle peut être choisie grâce à la fonction suivante :

\index{dateGenerator\_setStartDate}
\begin{verbatim}
void dateGenerator_setStartDate(struct dateGenerator_t * dateGen,
				motSimDate_t date);
\end{verbatim}

   Par défaut, cette date sera celle de la fin d'initialisation du
générateur, c'est-à-dire la date à laquelle sa loi d'inter-arrivée
aura été définie. En général, cela se fera à l'initialisation,
probablement avant de lancer la simulation, donc à la date 0.0.

%------------------------------------------------------------------------
%
%------------------------------------------------------------------------
\subsection{Choix de la loi d'inter arrivée}

\index{dateGenerator\_createPeriodic}
\begin{verbatim}
void dateGenerator_setRandomGenerator(struct dateGenerator_t * dateGen,
				      struct randomGenerator_t * randGen);
\end{verbatim}

%........................................................................
%
%........................................................................
\subsubsection{Les générateurs périodiques}

   On peut créer des générateurs périodiques grâce à la fonction
suivante, dont je ne pense pas avoir besoin de vous décrire l'unique
paramètre \ldots

\index{dateGenerator\_createPeriodic}
\begin{verbatim}
struct dateGenerator_t * dateGenerator_createPeriodic(double period);
\end{verbatim}

%........................................................................
%
%........................................................................
\subsubsection{Les générateurs Poissoniens}

   Ils sont créés par la fonction suivante, dont le paramêtre est
clair !

\index{dateGenerator\_createExp}
\begin{verbatim}
struct dateGenerator_t * dateGenerator_createExp(double lambda);
\end{verbatim}

%------------------------------------------------------------------------
%
%------------------------------------------------------------------------
\subsection{Obtention d'une date}

   Le rôle principal d'un générateur de date est donc de fournir des
dates. Voici la fonction qui permet cette merveille~:

\index{dateGenerator\_nextDate}
\begin{verbatim}
double dateGenerator_nextDate(struct dateGenerator_t * dateGen);
\end{verbatim}

%------------------------------------------------------------------------
%
%------------------------------------------------------------------------
\subsection{Fonctions annexes}

\index{dateGenerator\_isPeriodic}
\begin{verbatim}
int dateGenerator_isPeriodic(struct dateGenerator_t *  d);
\end{verbatim}


